\documentclass{article}
\usepackage{graphicx} % Required for inserting images
\usepackage[utf8]{inputenc}
\usepackage[T2A]{fontenc}
\usepackage[english, russian]{babel}
\usepackage{wrapfig}
\usepackage{listings}
\usepackage{hyperref}

\usepackage[left=20mm, top=15mm, right=15mm, bottom=45mm, nohead, footskip=10mm]{geometry}
 

\hypersetup{
    colorlinks=true,
    linkcolor=black,
    filecolor=magenta,      
    urlcolor=cyan
}



\usepackage{fancyhdr, fancybox}
\usepackage{amsmath, amsfonts, amssymb}

\usepackage{xcolor}

\definecolor{codegreen}{rgb}{0,0.6,0}
\definecolor{codegray}{rgb}{0.5,0.5,0.5}
\definecolor{codepurple}{rgb}{0.58,0,0.82}
\definecolor{backcolour}{rgb}{0.95,0.95,0.92}

\lstdefinestyle{mystyle}{
    backgroundcolor=\color{backcolour},   
    commentstyle=\color{codegreen},
    keywordstyle=\color{magenta},
    numberstyle=\tiny\color{codegray},
    stringstyle=\color{codepurple},
    basicstyle=\rmfamily\footnotesize,
    breakatwhitespace=false,         
    breaklines=true,                 
    captionpos=b,                    
    keepspaces=true,                 
    numbers=left,                    
    numbersep=5pt,                  
    showspaces=false,                
    showstringspaces=false,
    showtabs=false,                  
    tabsize=2
}
\lstset{style=mystyle}

\pagestyle{fancy}

\fancyhf{}
\fancyhead[L]{Лабораторная работа №3}
\fancyfoot[R]{\thepage}
\fancyfoot[L]{\hyperlink{tooc}{В содержание}}
\setcounter{page}{-1}

\headsep=10mm


% Custun commands
\newcommand{\tabulart}[1]{%
  \begin{tabular}{l}#1\end{tabular}
}

\newcommand{\sectionS}[1]{
    \section*{#1}
    \addcontentsline{toc}{section}{#1}
}
\newcommand{\subsectionS}[1]{
    \subsection*{#1}
    \addcontentsline{toc}{subsection}{#1}
}

\newcommand{\mathL}[3]{
\mathopen{}\left#1
    \begin{array}{l}#3\end{array}
\right#2 \mathclose{}
}

\newcommand{\andR}[0]{
    \text{ и }
}
\newcommand{\orR}[0]{
    \text{ или }
}
\newcommand{\ifR}[0]{
    \text{ если }
}


\newcommand{\nL}[1]{
    \quad\quad{\mathL{.}{.}{#1}}
}
\newcommand{\kw}[1]{
    \:\textbf{\underline{#1}}\:
}
\newcommand{\bo}[1]{
    \:\textbf{#1}\:
}
\newcommand{\es}[1]{
    \quad\mathL{.}{.}{#1}
}

\begin{document}

    
    \begin{center}
    \thispagestyle{empty}
    \textbf{"Федеральное государственное автономное образовательное учреждение
высшего образования
"Национальный исследовательский университет
"Высшая школа экономики"}
\\[10ex]
    Московский институт электроники и математики им. А.Н. Тихонова НИУ ВШЭ
Департамент компьютерной инженерии (или департамент электронной инженерии)
    \end{center}

\begin{tabular}{lr}
    \begin{array}{|l|l|l|l|l|}
    \hline
    \text{Раздел}           & \text{Max. оценка} & \tabulart{Итог.,\\оценка} & \tabulart{Итог.,\\оценка} & \tabulart{Итог.,\\оценка} \\ [1ex]\hline
    \text{Постановка}       	    & 0,5         &              &              &  \\ [4ex]\hline
    \text{Метод}                	    & 1           &              &              &              \\ [4ex]\hline
    \text{Спецификация} 	    & 0,5         &              &              &   \\ [4ex]\hline
    \text{Алгоритм}           	    & 1,5         &              &              &              \\ [4ex]\hline
    \text{Работа программы} & 1           &              &              &     \\ [4ex]\hline
    \text{Листинг}                   & 0,5         &              &              &      \\ [4ex]\hline
    \text{Тесты  }          	    & 1           &              &              &     \\ [4ex]\hline
    \text{Вопросы }         	    & 2           &              &              &         \\ [4ex]\hline
    \text{Доп. задание}   	    & 2           &              &              &       \\ [4ex]\hline
    \text{Итого }           	    & 10          &              &              &       \\ [4ex]\hline
    \end{array}

    
\tabulart{
 \quad\quad\quad\quad\quadОТЧЕТ\\
\;по лабораторной работе №3\\

Студент:\\
\;Пырлицану Никита\\
\;Евгеньевич\\

Группа: БИВ239\\

Вариант: №473 (3, 5, 4)\\

Руководитель: \\
\;Батонова Оксана Юрьевна\\

Оценка: \rule{2cm}{0.4pt}\\

Дата сдачи: \rule{2cm}{0.4pt}
}
\end{tabular}

\pagebreak
\thispagestyle{empty}
\hypertarget{tooc}{}
\tableofcontents{}


\pagebreak

\sectionS{Задания}
\quad I. 	Функция $y=f(x)$  задана таблицей:
\begin{tabular}{|c|c|c|c|}
    \hline
    $X_{i}$ & $X_1$ & $X_2$ & $X_3$ \\ \hline
    $Y_i$ & $Y_1$ & $Y_2$ & $Y_3$ \\
    \hline
\end{tabular}\\

Вычислить приближенное значение в точке по формуле:\\

\[
    f(x) = \left\{
    \begin{array}{c}
    
        y_1+\frac{x-x_1}{x_2-x_1}(y_2-y_1)\text{, если } x_1\leq x < x_2 \\

        y_2+\frac{x-x_2}{x_3-x_1}(y3-y2)\text{, если } x_2 \leq x < x_3 \\

        y_3\text{, если } x=x_3
    \end{array}
    \right.\\
\]


II. Вычислить значение по формуле:

\[
    H = \underset{j=1,m}{min}\:\underset{i=1,n}{max} |b_{i,j}| \text{, где } b_{i,j} \text{ - элементы матрицы B[0:n-1, 0:m-1]}
\]\\

III. Дан целочисленный массив A[1:n]. Написать программу, включающую
две функции с параметрами. В первой функции необходимо
подсчитать количество повторений каждого элемента массива A.
Вторая процедура решает следующую задачу: eдалить из массива A все неповторяющиеся элементы путем сдвига
(дополнительный массив не использовать).


\pagebreak


\sectionS{Постановка задачи}

\underline{Дано}

I. $x_1, x_2, x_3, y_1, y_2, y_3$ - вещ., $x$ - вещ.

II. lmax - нат., $n, m$ - нат., B[0:n-1, 0:m-1] - цел.

III. lmax - нат., $n$ - нат., A[0:n-1] - цел.\\
\underline{Результат}

I. $f(x)$ - вещ.

II. H - цел.

III. A[0:k-1] - цел. или <Nothing to delete!>\\
\underline{При}

I. $x_1 \leq x, x_2 \leq x_3$

II. $n, m \geq 1$ и $n, m < lmax$

III. $n \geq 1$ и $n \leq lmax$\\
\underline{Связь}\\\\
I. См. формулу в задании\\\\
II. См. формулу в задании
\begin{flalign*}
    \text{III. } &k\text{ - длинна массива A' (A после удаления повторов)}&\\
    &\forall i=\overline{0,n-1} \; \exists j=\overline{0,k-1} : 
    A[i] = A'[j]&\\
    &\forall i=\overline{0,k-1} \; \nexists j=\overline{0,n-1} : 
    i \neq j \andR A'[i] = A[j]&
\end{flalign*}


\pagebreak
\sectionS{Метод решения}

\subsectionS{Задание I.}

Метод функции f:
\[


\begin{array}{l}
    \text{вернуть }y_1+\frac{x-x_1}{x_2-x_1}(y_2-y_1)\text{, если } x_1\leq x < x_2 \\
    \text{вернуть }y_2+\frac{x-x_2}{x_3-x_1}(y3-y2)\text{, если } x_2 \leq x < x_3 \\
    \text{вернуть }y_3\text{, если } x=x_3
\end{array}
\]\\
\subsectionS{Задание II.}

Метод функции H:
\[


\begin{array}{l}
    min = MAX\_NUMBER\\
    max = 0\\[1ex]
    \left\{
        \begin{array}{l}
            \text{для } j=\overline{0,m-1}\\
            \left\{
                \begin{array}{l}
                    \text{для } i=\overline{0,m-1}\\
                    max = B[i, j], \text{ если } B[i, j] > max
                \end{array}
            \right.\\
            min = max, \text{ если } max > min
        \end{array}
    \right.\\
\end{array}
\]\\
\subsectionS{Задание III.}

\quad Метод функции findRep:

\mathL{.}{.}{
    bn = 0\\
    j = 0\\
    \mathL{\{}{.}{
        \text{для } i=\overline{0,n-1}\\
        \mathL{\{}{.}{
            \text{пока j < bn} \andR reps[j]\neq arr[i]\\
            j = j+1
        }\\
        
        reps[bn] = A[i]; repc[bn] = 1; bn = bn+1, \text{ если } j = bn\\
        repc[j] = repc[j] + 1 \text{ в противном случае}
    }
}
\\\\

Метод функции delNonRep:

\mathL{.}{.}{
    j=1\\
    \mathL{\{}{.}{
        \text{для } i=\overline{0,bn-1}\\
        \mathL{\{}{.}{
            \text{пока } a[j] \neq reps[i]\\
            j = j+1
        }, \ifR repc[i] = 1\\\\
        \mathL{\{}{.}{
            \text{для } l=j+1\\
            a[l-1] = a[l]
        }\\
        k = k-1
    }
}
\\\\

Метод решения:

\mathL{.}{.}{
    bn = findRep()\\
    delNonRep()
}


\pagebreak
\sectionS{Внешняя спецификация}
\subsectionS{Задание I}

\fbox{
\mathL{.}{.}{
    \text{Task I}.\\
}
}\\
\mathL{\{}{\}}{
    \fbox{<x_1> <x_2> <x_3> <y_1> <y_2> <y_3>}
}^\bigstar\text{ до }x_1 \leq x_2 \leq x_3\\
\[\\
\mathL{.}{.}{
    \mathL{\{}{\}}{
        \mathL{.}{.}{
            \fbox{\text{Enter x such that x_1 \leq x \leq x_3:}}\\
            \mathL{\{}{\}}{
                \fbox{<x>}
            }^\bigstar \text{ до } x_1 \leq x \leq x_3
        }\\[3ex]
        \fbox{f(\ll x \gg) = \ll f(x) \gg}
    }^\bigstar \text{ до } 1 \neq 1
}\]
\subsectionS{Задание II}

\fbox{
\mathL{.}{.}{
    \text{Task II.}\\
    \text{Enter n, m such that 1 }\leq n, m \leq \ll lmax \gg
}
}

\[


\mathL{\{}{\}}{
    \fbox{<n> <m>}
}^\bigstar \text{ до } 1 \leq n, m \leq lmax



\fbox{
\mathL{.}{.}{
    \text{Enter matrix }\ll n \gg\text{x}\ll m \gg:\\
    \text{Строка 1:}\\
    <B[1, 1]> <B[1, 2]> ... <B[1:m]>\\
    ...\\
    \text{Строка n:}\\
    <B[n, 1]> <B[n, 2]> ... <B[n, m]>\\
    H = \ll H() \gg
}
}
\]
\subsectionS{Задание III}\\

\fbox{Enter n greater than 1 and less than \ll lmax \gg:}
\[

\mathL{\{}{\}}{
    \fbox{<n>}
}^\bigstar \text{ до } 1 \leq n \leq lmax
\]\\

\fbox{\mathL{.}{.}{
Enter array with \ll n \gg elements:\\
<A[1]> ... <A[n]>\\
A = \{\ll A[1] \gg, ..., \ll A[k] \gg\}
}}

\pagebreak
\sectionS{Описание алгоритма на псевдокоде}
\subsectionS{Задание I}\\
\es{
    \kw{Алг} \text{<<Функция f>>}\\
    \kw{Вход} x_1, x_2, x_3, y_1, y_2, y_3, x - вещ.\\
    \kw{Выход} ret-вещ.\\
    \kw{нач}\\
    \es{
        \kw{если} x_1 \leq x\:\kw{и}\:x < x_2\:\kw{то}\\
        \es{
            ret := y_1+\frac{x-x_1}{x_2-x_1}(y_2-y_1)
        }\\
        \kw{всё}
        \kw{если} x_2 \leq x \kw{и} x < x_3 \kw{то}\\
        \es {
            ret := y_2+\frac{x-x_2}{x_3-x_1}(y3-y2)
        }\\
        \kw{всё}
        ret := y_3
        
    }\\
    \kw{кон}
}
\es{
    \kw{Алг} \text{«Лабораторная работа №3. Задание I»}\\
    \kw{нач}\\
    \es{
        \bo{вывод}(\text{<<Task I>>})\\
        \bo{вывод}(<<\text{Enter }x_1, x_2, x_3, y_1, y_2, y_3>>)\\
        \kw{цикл}\\
        \es{
            \bo{ввод}(x_1, x_2, x_3, y_1, y_2, y_3)\\
        }\\
        \kw{до} x_1 \leq x_2 \leq x_3\\
        \kw{цикл}\\
        \es{
            \bo{ввод}(x)
        }\\
        \kw{до} x_1 \leq x \leq x_3\\
        \bo{вывод}(\text{<<f(>>, x, <<=>>, f(x1, x2, x3, y1, y2, y3, x)})\\
    }\\
    \kw{кон}
}\\



\subsectionS{Задание II}


\es {
    \kw{Алг} \text{<<Функция H>>}\\
    \kw{Вход:}n, m, B\\
    \kw{Выход:} min\\
    \kw{нач}\\
    \es{
    min := MAX\_NUMBER\\
    \kw{цикл} \kw{от} j=0 \kw{до} m-1\\
    \es{
        max := B[0, j]\\
        \kw{цикл}\kw{от} i=0 \kw{до} n-1\\
        \es{
            \kw{если} |B[i, j]| > |max| \kw{то}\\
            \es{
                max := B[i, j]
            }\\
            \kw{всё}
        }\\
        \kw{кц}
        \kw{если} |max|<|min| \kw{то}\\
        \es{
            min = max
        }//
        \kw{всё}\\
    }\\
    \kw{кц}
    }\\
    \kw{кон}
}
\es {
\kw{Алг} \text{«Лабораторная работа №3. Задание II»}\\
    \kw{нач}\\
    \es{
        \bo{вывод}(\text{<<Task II>>})\\
        \bo{вывод}(\text{<<Enter n, m such that 1 }\leq n, m \leq>>, lmax)\\

        \kw{цикл}\\
        \es{
            \bo{ввод}(n, m)\\
        }\\
        \kw{до} 1 \leq n \leq lmax \kw{и} 1 \leq m \leq lmax\\

        \bo{вывод}(\text{<<Enter matrix >>, n, <<x>>, m})\\
        \bo{ввод}(B[0:n-1])\\
        \bo{вывод}(\text{<<H = >>, H(n, m, b)})\\
    }\\
    \kw{кон}
}

\pagebreak
\\
\subsectionS{Задание III}
\es {
    \kw{Алг}\text{ <<Функция findRep>>}\\
    \kw{Вход:}n, arr, reps, repc\\
    \kw{Выход:} bn\\
    \kw{нач}\\
    \:bn := 0\\
    \kw{цикл}\kw{от} i=0 \kw{до} n-1\\
    \es{
        j := 0\\
        \kw{цикл-пока} j<bn\kw{и}reps[j]\neq arr[i]\\
        \es{
            j := j+1
        }\\
        \kw{кц}\\
        \kw{если}j = bn\kw{то}\\
        \es{
            reps[bn] := arr[i]\\
            repc[bn] := 1\\
            bn := bn+1\\
        }\\
        \kw{иначе}\\
        \es{
            repc[j] := repc[j]+1
        }\\
        \kw{всё}\\
        
    }\\
    \kw{кон}
}
\es {
    \kw{Алг} \text{<<Функция delNonRep>>}\\
    \kw{Вход: } n, arr, cl\_arr\_len, reps, repc\\
    \kw{Выход:} n\\
    \kw{нач}\\
    \es{
        \:j := 0
        \kw{цикл}\kw{от} i=0 \kw{до} bn-1\\
        \es{
            \kw{если}repc[i]=1\kw{то}\\
            \es{
                \kw{цикл-пока} a[j] \neq reps[i]\\
                \es{
                    j := j+1
                }\\
                \kw{кц}\\
                \kw{цикл}\kw{от}l=j+1\kw{до}k-1\\
                \es{
                    a[l-1] := a[l]\\
                }
                \kw{кц}\\
                k := k-1
            }\\
            \kw{всё}\\
        }\\
    }\\
    \kw{кон}
}\\\\\\
\es {
\kw{Алг} \text{«Лабораторная работа №3. Задание III»}\\
    \kw{нач}\\
    \es{
        \bo{вывод}(\text{<<Task III.>>})\\
        \bo{вывод}(\text{<<Enter n greater than 0 and less then>>, lmax})\\
        \:k := n
        \bo{вывод}(\text{<<Enter array with>>, n, <<elements:>>})\\
        \:bn := findRep(n, a, reps, repc)\\
        \bo{вывод}(\text{<<A = [>>, A[0], A[1], ... A[k-1], <<]>>})\\
    }\\
    \kw{кон}
}

\pagebreak
\sectionS{Листинг программ}
\subsectionS{Задание I}
\begin{lstlisting}[language=C]
#include <stdio.h>
#include "safeIO.h"


float f(int x1, int x2, int x3, int y1, int y2, int y3, int x) {
    if (x1 <= x && x < x2) {
        return (float)y1 + ((float)(x-x1) / (x2-x1))*(y2-y1);
    }

    if (x2 <= x && x < x3) {
        return (float)y2 + ((float)(x-x2) / (x3-x1))*(y3-y2);
    }

    return (float)y3;
}


int main() {
    printf("Task I.\n");
    int x1, x2, x3, y1, y2, y3, x;
    
    printf("Enter x1, x2, x3, y1, y2, y3:\n");
    enterInts(6, &x1, &x2, &x3, &y1, &y2, &y3);
    
    for(;;) {
        printf("Enter x such that x1 <= x <= x3")
        enterIntRanged(&x, x1, x3);
        printf("f(%d) = %f", x, f(x1,x2,x3,y1,y2,y3, x));
    }
    

    return 0;
}
\end{lstlisting}\\
\subsectionS{Задание II}
\begin{lstlisting}[language=C]
#include <stdio.h>
#include <stdlib.h>
#include "safeIO.h"
#include <limits.h>

#define lmax 10

int H(int n, int m, int b[][lmax]) {
    int max,  min = INT_MAX;

    for (int *uj=*b; uj<(*b)+m; uj++) {
        max = *uj;
        for (int *ui=uj+lmax; ui<b[n-1]+m; ui+=lmax) {
            if (abs(*ui) > abs(max))
                max = *ui;
            printf("%d\n", *ui);
        }

        if (abs(max) < abs(min))
            min = max;
    }

    return abs(min);
}

int main() {
    printf("Task II.\n");
    int b[lmax][lmax];
    int n, m;

    printf("Enter n, m such that 1 <= n, m <= %d:\n", lmax);
    enterIntsRanged(2, 1, INT_MAX, &n, &m);

    printf("Enter matrix %dx%d:\n", n, m);
    for (int i=0; i<n; i++) {
        printf("String %d:\n", i+1);
        enterIntArray(b[i], m);
    }

    printf("H = %d", H(n, m, b));
}

\end{lstlisting}\\
\subsectionS{Задание III}
\begin{lstlisting}[language=C]
#include <stdio.h>
#include <safeIO.h>
#define lmax 10

int findRep(int n, int *arr, int* reps, int* repc) {
    int bn=0;
    int j;
    for (int i=0; i<n; i++) {
        for (j=0; j<bn && reps[j] != arr[i]; j++);

        if (j == bn) {
            reps[bn] = arr[i];
            repc[bn] = 1;
            bn++;
        }
        else {
            repc[j]++;
        }
    }

    return bn;
}

int delNonRep(int n, int *a, int cl_arr_len, int *reps, int *repc) {
    int j=0;
    for (int i=0; i<cl_arr_len; i++) {
        if (repc[i] == 1) {
            while(a[j] != reps[i]) j++;
            for (int l=j+1; l<n; l++) {
                a[l-1] = a[l];
            }
            n--;
        }
    }
    return n;
}


int main() {
    printf("Task III.\n");
    int a[lmax], reps[lmax], repc[lmax];
    int n;

    printf("Enter n greater than 0 and less then %d:\n", lmax);
    enterIntRanged(&n, 1, lmax);

    printf("Enter array with %d elements:\n", n);
    enterIntArray(a, n);

    int cl_arr_len = findRep(n, a, reps, repc);
    int k = delNonRep(n, a, cl_arr_len, reps, repc);
    
    if (k == n) {
    	printf("Nothing to delete!");
	return 0;
     }
    
    printf("A = ");
    printIntArray(k, a);

    return 0;
}
\end{lstlisting}\\
Файл "safeIO.h"
\begin{lstlisting}[language=C]
#include <stdio.h>
#include <stdarg.h>
#include <stdlib.h>
#include <limits.h>
#include <float.h>
#include <math.h>


void clearBuffer() {
    while(getchar() != '\n');
}

int* _enterInts(int count, int min_limit, int max_limit) {
    int *nums = malloc(sizeof(int)*count);
    float input;
    int inputCode;

    for (int i=0; i<count; i++) {
        inputCode = scanf("%f", &input);

        if (!inputCode || input < min_limit || input > max_limit || ((int)input) != input) {
            clearBuffer();
            printf("Wrong input! Enter %d more numbers!\n", count-i);
            i--;
        }
        else
            nums[i] = (int)input;
    }

    return nums;
}

void enterIntRanged(int *dir, int minLimit, int maxLimit) {
    int *nums = _enterInts(1, minLimit, maxLimit);
    *dir = nums[0];
    free(nums);
}

void enterInt(int *dir) {
    int *nums = _enterInts(1, INT_MIN, INT_MAX);
    *dir = nums[0];
    free(nums);
}

void enterInts(int count, ...) {
    va_list arg;
    va_start(arg, count);

    int *nums = _enterInts(count, INT_MIN, INT_MAX);
    for (int i=0; i<count; i++)
        *va_arg(arg, int*) = nums[i];
    free(nums);
    va_end(arg);
}

void enterIntsRanged(int minLimit, int maxLimit, int count, ...) {
    va_list arg;
    va_start(arg, count);
    int *nums = _enterInts(count, minLimit, maxLimit);
    for (int i=0; i<count; i++)
        *va_arg(arg, int*) = nums[i];
    free(nums);
    va_end(arg);
}

void enterIntArrayRanged(int *ptr, int minLimit, int maxLimit, int len) {
    int *nums = _enterInts(len, minLimit, maxLimit);
    for (int i=0; i<len; i++)
        ptr[i] = nums[i];
    free(nums);

}

void enterIntArray(int *ptr, int len) {
    enterIntArrayRanged(ptr, len, INT_MIN, INT_MAX);
}



float *_enterFloats(int count, float min_limit, float max_limit) {
    float *nums = malloc(sizeof(float)*count);
    double input;
    int inputCode;

    for (int i=0; i<count; i++) {
                inputCode = scanf("%lf", &input);

        if (!inputCode || input < min_limit || input > max_limit) {
            clearBuffer();
            printf("Wrong input! Enter %d more numbers:\n", count-i);
            ;
            i--;
        }
        else
            nums[i] = (float)input;
    }
    printf("RETURN!");

    return nums;
}


void enterFloatRanged(float *ptr, float min_val, float max_val) {
    float *nums = _enterFloats(1, min_val, max_val);
    *ptr = nums[0];
    free(nums);
}

void enterFloat(float *ptr) {
    float *nums = _enterFloats(1, FLT_MIN, FLT_MAX);
    *ptr = nums[0];
    free(nums);
}

void enterFloatsRanged(float min_val, float max_val, int count, ...) {
    va_list arg;
    va_start(arg, count);
    float *nums = _enterFloats(count, min_val, max_val);

    for (int i=0; i<count; i++) {
        float *ptr = va_arg(arg, float*);
        *ptr = nums[i];
    }

    free(nums);
    va_end(arg);
}

void enterFloats(int count, ...) {
    va_list arg;
    va_start(arg, count);
    float *nums = _enterFloats(count, FLT_MIN, FLT_MAX);

    for (int i=0; i<count; i++) {
        float *ptr = va_arg(arg, float*);
        *ptr = nums[i];
    }

    free(nums);
    va_end(arg);
}


void printIntArray(int len, int *arr) {
    printf("{");
    if (len < 1) {
        printf("}");
        return;
    }
    for (int i=0; i<len-1; i++)
        printf("%d, ", arr[i]);
    printf("%d}", arr[len-1]);
}
\end{lstlisting}

\pagebreak
\sectionS{Тесты}
\begin{array}{ll}
\es{
\textbf{Задание I}\\
\begin{array}{|c|c|c|}
    \hline
    \text{№} & \text{Входные данные} & \text{Выходные данные}\\\hline
    1 & \tabulart{x_1=1\;x_2=5\;x_3=10\\y_1=1\;y_2=2\;y_3=3\\x=1} & f(1) = 1.000000\\\hline
    2 & \tabulart{x_1=1\;x_2=5\;x_3=10\\y_1=1\;y_2=2\;y_3=3\\x=2} & f(2) = 1.250000\\\hline
\end{array} 
} &
\es{
\textbf{Задание II}\\
\begin{array}{|c|c|c|}
    \hline
    \text{№} & \text{Входные данные} & \text{Выходные данные}\\\hline
    1 & \tabulart{n=2,\;m=2\;\\B=\;[\;5,\;2]\\\;\;\;\;\;[10, 1]} & H = 2\\\hline
\end{array} \\
}\\[15ex]
\es{
\textbf{Задание III}\\
\begin{array}{|c|c|c|}
    \hline
    \text{№} & \text{Входные данные} & \text{Выходные данные}\\\hline
    1 & \tabulart{n=5\\A=[1, 2, 3,4,5]} & A=\{\}\\\hline
    2 & \tabulart{n=5\\A=[1, 1, 2, 2, 3]} & A=\{1, 1, 2, 2\}\\\hline
    3 & \tabulart{n=4\\A=[1, 1, 2,  2]} & \text{Nothing to change!}\\\hline
\end{array} 
}
\end{array}\\


\end{document}
