\documentclass{article}
\usepackage{graphicx} % Required for inserting images
\usepackage[utf8]{inputenc}
\usepackage[T2A]{fontenc}
\usepackage[english, russian]{babel}
\usepackage{wrapfig}
\usepackage{listings}
\usepackage{hyperref}

\usepackage[left=20mm, top=15mm, right=15mm, bottom=45mm, nohead, footskip=10mm]{geometry}
 

\hypersetup{
    colorlinks=true,
    linkcolor=black,
    filecolor=magenta,      
    urlcolor=cyan
}



\usepackage{fancyhdr, fancybox}
\usepackage{amsmath, amsfonts, amssymb}

\usepackage{xcolor}

\definecolor{codegreen}{rgb}{0,0.6,0}
\definecolor{codegray}{rgb}{0.5,0.5,0.5}
\definecolor{codepurple}{rgb}{0.58,0,0.82}
\definecolor{backcolour}{rgb}{0.95,0.95,0.92}

\lstdefinestyle{mystyle}{
    backgroundcolor=\color{backcolour},   
    commentstyle=\color{codegreen},
    keywordstyle=\color{magenta},
    numberstyle=\tiny\color{codegray},
    stringstyle=\color{codepurple},
    basicstyle=\rmfamily\footnotesize,
    breakatwhitespace=false,         
    breaklines=true,                 
    captionpos=b,                    
    keepspaces=true,                 
    numbers=left,                    
    numbersep=5pt,                  
    showspaces=false,                
    showstringspaces=false,
    showtabs=false,                  
    tabsize=2
}
\lstset{style=mystyle}

\pagestyle{fancy}

\fancyhf{}
\fancyhead[L]{Лабораторная работа №3}
\fancyfoot[R]{\thepage}
\fancyfoot[L]{\hyperlink{tooc}{В содержание}}
\setcounter{page}{-1}

\headsep=10mm


% Custun commands
\newcommand{\tabulart}[1]{%
  \begin{tabular}{l}#1\end{tabular}
}

\newcommand{\sectionS}[1]{
    \section*{#1}
    \addcontentsline{toc}{section}{#1}
}
\newcommand{\subsectionS}[1]{
    \subsection*{#1}
    \addcontentsline{toc}{subsection}{#1}
}

\newcommand{\mathL}[3]{
\mathopen{}\left#1
    \begin{array}{l}#3\end{array}
\right#2 \mathclose{}
}

\newcommand{\andR}[0]{
    \text{ и }
}
\newcommand{\orR}[0]{
    \text{ или }
}
\newcommand{\ifR}[0]{
    \text{ если }
}


\newcommand{\nL}[1]{
    \quad\quad{\mathL{.}{.}{#1}}
}
\newcommand{\kw}[1]{
    \:\textbf{\underline{#1}}\:
}
\newcommand{\bo}[1]{
    \:\textbf{#1}\:
}
\newcommand{\es}[1]{
    \quad\mathL{.}{.}{#1}
}

\begin{document}

    
    \begin{center}
    \thispagestyle{empty}
    \textbf{"Федеральное государственное автономное образовательное учреждение
высшего образования
"Национальный исследовательский университет
"Высшая школа экономики"}
\\[10ex]
    Московский институт электроники и математики им. А.Н. Тихонова НИУ ВШЭ
Департамент компьютерной инженерии (или департамент электронной инженерии)
    \end{center}

\begin{tabular}{lr}
    \begin{array}{|l|l|l|l|l|}
    \hline
    \text{Раздел}           & \text{Max. оценка} & \tabulart{Итог.,\\оценка} & \tabulart{Итог.,\\оценка} & \tabulart{Итог.,\\оценка} \\ [1ex]\hline
    \text{Работа программы} & 1           &              &              &     \\ [4ex]\hline
    \text{Тесты}               & 0,5         &              &              &  \\ [4ex]\hline
    \text{Правильность алгоритма}                        & 1           &              &              &              \\ [4ex]\hline
    \text{Ответы на вопросы}         & 0,5         &              &              &   \\ [4ex]\hline
    \text{Алгоритм}                 & 1,5         &              &              &              \\ [4ex]\hline
    \text{Доп. задание} & 1           &              &              &     \\ [4ex]\hline
    \end{array}

    
\tabulart{
 \quad\quad\quad\quad\quadОТЧЕТ\\
\;по лабораторной работе №3\\

Студент:\\
\;Пырлицану Никита\\
\;Евгеньевич\\

Группа: БИВ239\\

Вариант: №473 (6, 4, 1)\\

Руководитель: \\
\;Батонова Оксана Юрьевна\\

Оценка: \rule{2cm}{0.4pt}\\

Дата сдачи: \rule{2cm}{0.4pt}
}
\end{tabular}

\pagebreak
\thispagestyle{empty}
\hypertarget{tooc}{}
\tableofcontents{}


\pagebreak

\sectionS{Задания}

Дано k символьных строк. Каждая строка содержит латинские и русские
буквы, цифры, а также все возможные разделители. 

Требуется:

1) Выделить из каждой строки и напечатать подстроки ограниченные с обеих сторон одной или несколькими латинскими
буквами

2) Среди выделенных подстрок найти подстроку (если таких подстрок
несколько, выбирается первая из них): содержащую максимальное (но не нулевое) число цифр

3) Преобразовать исходную строку, которой принадлежит найденная
подстрока, следующим образом: удалить путем сдвига все латинские буквы;

\pagebreak



\sectionS{Листинг программ}

\pagebreak
\sectionS{Тесты}
% \tabulart{} & \tabulart{}\\\hline
\begin{array}{|c|c|c|}
    \hline
    \text{№} & \text{Входные данные} & \text{Выходные данные}\\\hline
    1 & \tabulart{k=1\\a123b} & \tabulart{I: a123b \\ II: a123b \\ III: 123}\\\hline
    2 & \tabulart{k=1\\} & \tabulart{I. Подходящих подстрок не найдено!}\\\hline
    3 & \tabulart{k=1\\a123b456c12} & \tabulart{I: 1. a123b\\\quad2. a123b456c\\\quad3. b456c\\II: a123b456c\\III: 12345612}\\\hline
    4 & \tabulart{k=3\\a123b\\g45678c\\z12} & \tabulart{I: 1. a123b\\\quad 2. g45678c\\II: g45678c\\III: 45678}\\\hline
    5 & \tabulart{k=1\\dваавb} & \tabulart{I: dваавb\\II. Поддходящей подстроки не найдено!}\\\hline
\end{array} 



\end{document}
